\documentclass{article}
\usepackage[margin=1in]{geometry}
\usepackage[brazil]{babel}
\usepackage[utf8]{inputenc}
\usepackage{amsmath}
\usepackage{indentfirst}
\usepackage{cite}
\usepackage{url}
\usepackage{graphics}
\usepackage{graphicx}
\usepackage{caption}
\usepackage{subcaption}
\usepackage{multirow}
\usepackage{amsfonts}
\usepackage{color}
\usepackage{listings}

\title{Introdução à Compilação\\Entrega 3}
\author{Bianca Madoka Shimizu Oe\\
		Luis Fernando Dorelli de Abreu\\
		Rafael Umino Nakanishi}
\date{}

\definecolor{gray}{RGB}{124,124,124}
\lstset{language=Pascal,numbers=left,stepnumber=1,frame=bottom|top,tabsize=2,breaklines=true,basicstyle=\small\ttfamily,
		keywordstyle=\color{blue}\bfseries,commentstyle=\color{gray}, numbersep=-8pt}

\begin{document}
\maketitle

\section{Objetivo} % (fold)
\label{sec:objetivo}
	O objetivo dessa parte final do projeto da disciplina é a implementação do analisador semântico para o compilador para a linguagem \verb=LALG= disponibilizada para os alunos e a geração de código intermediário. A seguir serão discutidos a abordagem utilizada para lidar com a tabela de símbolos e os tratamentos de erros que são gerados pelo compilador. Em seguida estão listados o meio para compilar o código e o método de executá-lo e alguns exemplos que foram utilizados para testar a compilação e suas respectivas saídas.

% section objetivo (end)

\section{Decisões de projeto} % (fold)
\label{sec:decisoes_de_projeto}
	\subsection{Implementação} % (fold)
	\label{sub:implementacao}
		A implementação do compilador para a linguagem \verb=LALG= foi feita utilizando o \verb=yacc= (\em Yet Another Compiler-Compiler\normalfont) em linguagem de programação \verb=C=. Essa decisão foi influenciada principalmente devido à possibilidade de continuidade a partir da entrega anterior. 

		Para armazenar os atributos, utilizamos o recurso proporcionado pelo \verb=yacc=, que consiste na definição de um tipo, que pode ser uma estrutura. Para acessá-los, utilizamos o símbolo reservado \$. Uma variável desse tipo é criada sempre que um token é obtido pelo analizados léxico.
		
	% subsubsection implementacao (end)

	\subsection{Tabela de símbolos} % (fold)
	\label{sub:tabela_de_simbolos}
		A tabela de símbolos utilizada nessa implementação é baseado na estrutura de dados \emph{trie}. Nesse caso, cada nó da árvore armazena um caracter correspondente ao nome do identificador. Ao se chegar ao fim da cadeia de caracteres, o nó correspondente armazena os atributos do identificador:

		\begin{itemize}
			\item \verb=type=: tipo do identificador (\verb=integer= ou \verb=real=)
			\item \verb=category=: categoria a que pertence (constante ou variável)
			\item \verb=scope=: escopo do identificador
			\item \verb=ival=: valor da variável caso seja um número inteiro
			\item \verb=rval=: valor da variável caso seja um número real
			\item \verb=paramQty=: quantidade de parâmetros, no caso de procedimentos
			\item \verb=line=: linha em que foi declarado
			\item \verb=parameters=: lista de parâmetros, no caso de procedimentos
			\item \verb=name=: nome completo do identificador, com um máximo de 20 caracteres
			\item \verb=codeLine=: auxilia na resolução de desvios durante a geração de código intermediário
			\item \verb=address=: endereço da variável no programa
			\item \verb=parametros=: Ponteiro inicial para uma lista ligada contendo entradas de tabela de símbolos para os parâmetros (no caso de um procedimento)
		\end{itemize}
	% subsection tabela_de_s_mbolos (end)

	Para resolver problemas de escopo, foram criadas diversas tabelas de símbolos separadas, uma para cada escopo. Quando o analisador volta de um escopo (retorna da declaração de um procedimento), deleta-se a tabela de símbolos do escopo anterior. 

	\subsection{Tratamento de erros} % (fold)
	\label{ssub:tratamento_de_erros}
		Os seguintes erros são tratados pela análise semântica.

		\begin{itemize}
			\item Variável ou procedimento não declarado
			\item Variável ou procedimento declarado mais de uma vez
			\item Incompatibilidade de parâmetros formais e reais: número, ordem e tipo
			\item Atribuição de um real a um inteiro
			\item Divisão entre números não inteiros
			\item \verb=readln= e \verb=writeln= com parâmetros de tipo diferentes
		\end{itemize}
	
	% subsubsection tratamento_de_erros (end)

	\subsection{Geração de código} % (fold)
	\label{sub:geracao_de_codigo}
		A geração de código pelo compilador é realizada em paralelo com a análise semântica e é interrompida caso seja encontrado algum erro de natureza léxica, sintática e/ou semântica.

		Consideramos que o código inicia na linha 0 e os códigos que são possivelmente gerados podem ser encontrados no Quadro \ref{tab:codigos}. Todos os comando são os mesmos vistos em sala de aula, ao longo da disciplina.

		\begin{table}[h]
			\centering	
			\ttfamily
			\begin{tabular}{c c c c c}
				\hline
				CRCT & CRVL & SOMA & SUBT & MULT \\
				INVE & CONJ & DISJ & NEGA & CPME \\
				CPMA & CPIG & CDES & CPMI & CMAI \\
				ARMZ & DSVI & DSVF & LEIT & IMPR \\
				ALME & INPP & PARA & PUSHER & CHPR \\
				DESM & RTPR & COPVL & PARAM & DIVI \\
				\hline
			\end{tabular}
			\caption{Códigos possivelmente gerados pela geração de código}
			\label{tab:codigos}
		\end{table}
	
	Observe que a tabela de símbolos implementada não armazena os valores das constantes, mas apenas substitui os mesmos no código a ser gerado.
	% subsection geracao_de_codigo (end)

% section decis_es_de_projeto (end)

\section{Compilação} % (fold)
\label{sec:compilacao}
	Para compilação do projeto em sistemas \emph{UNIX}, execute o arquivo \verb=Makefile= disponibilizado por meio do comando \verb=make=. 

	Para o sistema operacional \emph{Windows}, execute a seguinte sequência de comandos:

	\begin{center}
		\begin{minipage}[ht]{0.5\textwidth}
			\begin{verbatim}
				yacc -d -v yacc.yacc --verbose 
				flex lalg.l
				gcc lex.yy.c trie.c symbolTable.c -lfl -o t3.exe
			\end{verbatim}
		\end{minipage}
	\end{center}

	Para execução do analisador sintático utilize os comandos:

	\begin{center}
		\begin{minipage}[ht]{0.5\textwidth}
			\begin{verbatim}
				t3.exe <nome_do_arquivo> <nome_do_arquivo_bytecode>
			\end{verbatim}
		\end{minipage}
	\end{center}

	Observe que o analisador só termina quando encontra um \emph{end-of-file}. Caso o arquivo de entrada esteja gramaticamente correto, obtém-se como saída um arquivo com código intermediário referente ao código dado como entrada denominado ``\emph{youGiveMe.noName}'' caso o usuário não entre com algum nome para tal arquivo.
% section compila_o (end)

\section{Exemplos} % (fold)
\label{sec:exemplos}
	
	\subsection{Exemplo 1} % (fold)
	\label{sub:exemplo_1}
		Esse exemplo está gramaticamente correto. E por sinal é o código fornecido na oitava avaliação da disciplina. 
		O programa foi compilado com sucesso e obteve um código intermediário.

		\subsubsection*{Entrada} % (fold)
		\label{ssub:entrada}
		
		\begin{lstlisting}
			program fatorial;
			var n: integer;
			procedure fat(x : integer);
			var resultado: integer;
			begin
				resultado := 1;
				while x > 1 do
				begin
					resultado := resultado * x;
					x := x - 1;
				end;

				writeln(resultado);
			end;

			begin
				readln(n);
				fat(n);
			end.
		\end{lstlisting}

		% subsubsection entrada (end)

		\subsubsection*{Saída gerada} % (fold)
		\label{ssub:sa_da_gerada}
			\begin{center}
			\begin{minipage}[t]{0.4\textwidth}
			\begin{lstlisting}[firstnumber=0]
			INPP
			ALME1
			DSVI 24
			COPVL
			ALME1
			CRCT 1
			ARMZ 3
			CRVL 2
			CRCT 1
			CPMA
			DSVF 20
			CRVL 3
			CRVL 2
			MULT
			ARMZ 3
			\end{lstlisting}
			\end{minipage}
			\begin{minipage}[t]{0.4\textwidth}
			\begin{lstlisting}[firstnumber=15]
			CRVL 2
			CRCT 1
			SUBT
			ARMZ 2
			DSVI 7
			CRVL 3
			IMPR
			DESM 2
			RTPR
			LEIT
			ARMZ 0
			PUSHER 29
			PARAM 0
			CHPR 3
			PARA
			\end{lstlisting}
			\end{minipage}
			\end{center}
		% subsubsection sa_da_gerada (end)
	% subsection exemplo_1 (end)

	\subsection{Exemplo 2} % (fold)
	\label{sub:exemplo_2}

		\subsubsection*{Entrada} % (fold)
		\label{ssub:entrada}
		\begin{lstlisting}
			prog prog; { program escrito errado }
			{ declaracao de variaveis }
			const pi=3.14; { constante declarada depois de variavel }
			var x: integer;
			var y:real;
			{ declaracao de procedimento }
			procedure proc1 (x:real, y:real);
			begin
				readln(x,y);
				if x > 3 then
					x := 3
				else
					x := 4 { faltou ; }
			end;
			procedure proc2 (x:integer);
			begin
				writeln(x);
				while x/y > 3 do
					x := 3x + 2;
			end;
			begin
				proc1 (x; y);
				proc2;
			end.
		\end{lstlisting}
		% subsubsection* entrada (end)
		\subsubsection*{Saída gerada} % (fold)
		\label{ssub:sa_da_gerada}
		\begin{lstlisting}[language=TeX]
		Erro na linha 1: 'identificador' inesperado [prog], esperava 'program'
		Erro na linha 3: '=' inesperado, esperava ':='
		Erro na linha 7: ',' inesperado, esperava ';' ou ')'
		Erro na linha 9: tipos diferentes no comando readln.
		Erro na linha 14: 'end' inesperado, esperava ';'
		Erro na linha 18: Divisao de numeros nao inteiros.
		Erro na linha 19: 'Numero_mal_formado' inesperado [3x], esperava 'identificador' ou '(' ou 'numero_inteiro' ou 'numero_real'
		Erro na linha 22: Parametro 1 do procedimento proc1:
			Inteiro obtido, Real esperado. 
		Erro na linha 23: Falta de parametros na chamada do procedimento proc2. 1 parametro	
		\end{lstlisting}
		% subsubsection sa_da_gerada (end)
	% subsection exemplo_2 (end)

	\subsection{Exemplo 3} % (fold)
	\label{sub:exemplo_3}
		\subsubsection*{Entrada} % (fold)
		\label{ssub:entrada}
		\lstinputlisting{../Testes/Entrada/teste1.pas}
		\subsubsection*{Saída} % (fold)
		\label{ssub:sa_da}
		\begin{center}
		\begin{minipage}[t]{0.3\textwidth}
		\begin{lstlisting}[firstnumber=0]
		INPP
		ALME1
		ALME1
		ALME1
		DSVI 59
		ALME1
		LEIT
		ARMZ 0
		CRVL 0
		CRCT 3
		CPMA
		DSVF 17
		CRCT 3.570000
		ARMZ 1
		CRCT 3
		ARMZ 0
		DSVI 36
		CRVL 0
		CRCT 3
		CPME
		DSVF 36
		CRVL 0
		CRVL 0
		CRCT 2
		SOMA
		CRCT 3
		\end{lstlisting}
		\end{minipage}	
		\begin{minipage}[t]{0.3\textwidth}
		\begin{lstlisting}[firstnumber=26]
		DIVI 0
		SOMA
		ARMZ 0
		CRCT 3.570000
		CRCT 3
		DIVI 0
		CRVL 0
		SOMA
		ARMZ 0
		DSVI 17
		CRVL 4
		IMPR
		DESM 1
		RTPR
		COPVL
		COPVL
		COPVL
		ALME1
		CRVL 4
		CRVL 5
		SOMA
		ARMZ 7
		CRVL 7
		CRCT 1
		SUBT
		ARMZ 6
		\end{lstlisting}
		\end{minipage}
		\begin{minipage}[t]{0.3\textwidth}
		\begin{lstlisting}[firstnumber=52]
		CRVL 7
		CRCT 10
		CPMA
		DSVF 57
		DSVI 44
		DESM 4
		RTPR
		PUSHER 61
		CHPR 5
		PUSHER 66
		PARAM 0
		PARAM 0
		PARAM 1
		CHPR 40
		CRVL 0
		CRVL 1
		CRCT 3
		CRVL 0
		MULT
		SOMA
		MULT
		CRCT 5
		DIVI 0
		ARMZ 0
		PARA
		\end{lstlisting}
		\end{minipage}
		\end{center}
		% subsubsection* entrada (end)
	% subsection exemplo_3 (end)
% section exemplos (end)

\end{document}

\documentclass{article}
\usepackage[margin=1in]{geometry}
\usepackage[brazil]{babel}
\usepackage[utf8]{inputenc}
\usepackage{amsmath}
\usepackage{indentfirst}
\usepackage{cite}
\usepackage{url}
\usepackage{graphics}
\usepackage{graphicx}
\usepackage{caption}
\usepackage{subcaption}
\usepackage{multirow}
\usepackage{amsfonts}
\usepackage{color}
\usepackage{listings}

\title{Introdução à Compilação\\Entrega 3}
\author{Bianca Madoka Shimizu Oe\\
		Luis Fernando Dorelli de Abreu\\
		Rafael Umino Nakanishi}
\date{}

\definecolor{gray}{RGB}{124,124,124}
\lstset{language=Pascal,numbers=left,stepnumber=1,frame=bottom|top,tabsize=2,breaklines=true,basicstyle=\small\ttfamily,
		keywordstyle=\color{blue}\bfseries,commentstyle=\color{gray}, numbersep=-7pt}

\begin{document}
\maketitle

\section{Objetivo} % (fold)
\label{sec:objetivo}
	O objetivo dessa parte final do projeto da disciplina é a implementação do analisador semântico para o compilador para a linguagem \verb=LALG= disponibilizado para os alunos. A seguir serão discutidos as mudanças que foram necessárias para a implementação do projeto. Nas seções seguintes mostraremos a abordagem utilizada para lidar com a tabela de símbolos e os tratamentos de erros que são gerados pelo compilador. Por fim, estão listados o meio para compilar o código e o método de executá-lo e alguns exemplos que foram utilizados para testar a compilação e suas respectivas saídas.

% section objetivo (end)

\section{Mudanças realizadas} % (fold)
\label{sec:mudancas_realizadas}
	Sinto muito. Não sei.

	Próximo.
% section mudan_as_realizadas (end)

\section{Decisões de projeto} % (fold)
\label{sec:decisoes_de_projeto}
	\subsection{Implementação} % (fold)
	\label{ssub:implementacao}
		A implementação do compilador para a linguagem \verb=LALG= foi feita utilizando o \verb=yacc= (\em Yet Another Compiler-Compiler\normalfont) em linguagem de programação \verb=C=. Essa decisão foi influenciada principalmente devido à possibilidade de continuidade a partir da entrega anterior. Para a implementação da tabela de símbolos utilizada pela análise semântica foi utilizado a estrutura de dados \emph{trie} e seus detalhes podem ser encontrados na seção \ref{sub:tabela_de_simbolos}.

		((Falar sobre os pseudo-atributos))
	
	% subsubsection implementa_o (end)

	\subsection{Tabela de símbolos} % (fold)
	\label{sub:tabela_de_simbolos}
		A tabela de símbolos utilizada nessa implementação é baseado na estrutura de dados \emph{trie}. Nesse caso, cada nó da árvore armazena um caracter correspondente ao nome do identificador. Ao se chegar ao fim do da cadeia de caracteres, o nó correspondente armazena os atributos do identificador:

		\begin{itemize}
			\item \verb=type=: tipo do identificador
			\item \verb=category=: categoria a que pertence
			\item \verb=scope=: escopo do identificador
			\item \verb=ival=: valor carregado pelo identificador
			\item \verb=paramQty=: quantidade de parâmetros, no caso de procedimentos
			\item \verb=rval=: valor de retorno (?)
			\item \verb=line=: linha em que foi declarado
			\item \verb=parameters=: lista de parâmetros, no caso de procedimentos
			\item \verb=name=: nome completo do identificador, com um máximo de 20 caracteres
		\end{itemize}
	% subsection tabela_de_s_mbolos (end)

	\subsection{Tratamento de erros} % (fold)
	\label{ssub:tratamento_de_erros}
		Os seguintes erros são tratados pela análise semântica.

		\begin{itemize}
			\item Variável ou procedimento não declarado
			\item Variável ou procedimento declarado mais de uma vez
			\item Incompatibilidade de parâmetros formais e reais: número, ordem e tipo
			\item Uso de variáveis de escopo inadequado
			\item Atribuição de um real a um inteiro
			\item Divisão que entre números não inteiros
			\item \verb=readln= e \verb=writeln= com variáveis de tipo diferentes
		\end{itemize}
	
	% subsubsection tratamento_de_erros (end)

	\subsection{Geração de pseudocódigo} % (fold)
	\label{sub:geracao_de_pseudocodigo}
		A geração de código pelo compilador é realizado em paralelo com a análise semântica e é interrompido caso seja encontrado algum erro de natureza léxica, sintática e/ou semântica.

	% subsection geracao_de_pseudocodigo (end)

% section decis_es_de_projeto (end)

\section{Compilação} % (fold)
\label{sec:compilacao}
	Para compilação do projeto em sistemas \emph{UNIX}, execute o arquivo \verb=Makefile= disponibilizado por meio do comando \verb=make=. Para o sistema operacional \emph{Windows}, execute a seguinte sequência de comandos:

	\begin{center}
		\begin{minipage}[ht]{0.5\textwidth}
			\begin{verbatim}
				yacc -d -v yacc.yacc --verbose 
				flex lalg.l
				gcc lex.yy.c trie.c symbolTable.c -lfl -o t3.exe
			\end{verbatim}
		\end{minipage}
	\end{center}

	Para execução do analisador sintático utilize os comandos:

	\begin{center}
		\begin{minipage}[ht]{0.5\textwidth}
			\begin{verbatim}
				t3.exe <nome_do_arquivo>
			\end{verbatim}
		\end{minipage}
	\end{center}

	Observe que o analisador só termina quando encontra um \emph{end-of-file}. Caso o arquivo de entrada esteja gramaticamente correto, obtém-se como saída um arquivo com o \emph{bytecode} referente ao código dado como entrada denominado \emph{code.code}.
% section compila_o (end)

\section{Exemplos} % (fold)
\label{sec:exemplos}

% section exemplos (end)

\end{document}

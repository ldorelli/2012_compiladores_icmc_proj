\documentclass {article}
\usepackage[margin=1in]{geometry}
\usepackage[brazil]{babel}
\usepackage[utf8]{inputenc}
\usepackage{amsmath}
\usepackage{indentfirst}
\usepackage{cite}
\usepackage{multirow}
\usepackage{verbatim}
\usepackage{listings}
\usepackage{booktabs}
\usepackage{subfig}
\usepackage{graphics}
\usepackage{graphicx}

\title {Trabalho 2}
\author {Bianca Madoka Shimizu Oe \\ 
		Luís Fernando Dorelli de Abreu \\  
		Rafael Umino Nakanishi}

\begin{document}

\titlepage
\maketitle

\section{Objetivo}
	Neste relatório descreveremos a utilização do analisador sintático para a LALG. Apresentaremos as pricipais decisões de projetos e as respectivas justificativas. A seguir uma breve explicação sobre o analisador utilizado e alguns exemplos que foram usados para fins de teste.

	Esse relátório está organizado da seguinte forma. Na Seção \ref{section:lex}, são discutidas as mudanças no analisador léxico implementadas para fornecer suporte ao analisador sintático. Na Seção \ref{section:decisoes}, são apresentadas as decisões de projeto e detalhes de implementação relacionadas ao analisador sintático produzido. Na Seção \ref{section:compilacao}, encontram-se as instruções para compilação e execução do analisador sintático. Por fim, na Seção \ref{section:exemplos}, são apresentados exemplos de entrada (programas LALG possivelmente incorretos) e suas respectivas saídas.

\section{Mudanças Realizadas}\label{section:lex}

	Para dar suporte ao analisador sintático, foi adicionado ao analisador léxico da etapa anterior um contador de linhas, de forma que, quando um erro sintático for encontrado, seja possível mostrar a linha exata onde o erro ocorreu. A contagem de linhas é independente da presença de comentários: as linhas de comentário também são computadas.

	Também em relação a entrega anterior, foi adicionada a análise léxica a detecção de comentários não fechados. Nesse caso, a contagem de linhas é parada, para que, quando o erro for detectado, seja mostrada a primeira linha do comentário, e não a última. 

\section{Decisões de Projeto}\label{section:decisoes}
	\subsection{Implementação}
		Para implementação de um analisador sintático para a linguagem LALG, optamos pela utilização do YACC (\emph{Yet Another Compiler Compiler}). Essa decisão foi influenciada pela simplicidade e elegância habilitadas pela ferramenta, especialmente no que diz respeito a descrição da gramática.  

		A implementação do analisador supracitado é dividida em três seções: declarações, regras e programas, separadas por dois símbolos de porcentagem (\verb=%%=). A seguir são discutidos o conteúdo implementado em cada seção. 


		\subsubsection{Declarações}
			Nessa seção são declarados todos os tokes utilizados pelo analisador sintático. Todos os tokens declarados são importados da fase anterior do projeto e podem ser vistas nos Quadros \ref{table:op}, \ref{table:simb} e \ref{table:reserv}. 

			Para declaração utilizamos o recurso do YACC \verb=%token=, seguido dos nomes dos tokes e um nome semântico para os mesmos. Esse nome semântico é substituído nas mensagens de erros para uma apresentação mais amigável ao usuário.

			\begin{table}
				\begin{minipage}[ht]{0.5\textwidth}
					\centering
					\begin{tabular}{rcc}
						\toprule
						Nome do operador & Token & Símbolo\\
						\hline
						\hline
							Atribuição & op\_at & := \\
							Igualdade & op\_eq & = \\ 
							Diferença &  op\_df & $<>$ \\ 
							Maior ou igual & op\_ge & $>=$ \\ 
							Menor ou igual & op\_le & $<=$ \\ 
							Maior & op\_gr & $>$ \\
							Menor &op\_ls & $<$ \\ 
							Soma & op\_pl & + \\ 
							Subtração & op\_mi & - \\ 
							Multiplicação & op\_ml & * \\
							Divisão & op\_dv & / \\
						\bottomrule
					\end{tabular}
					\caption{Quadro de operadores}
					\label{table:op}
				\end{minipage}
				~
				\begin{minipage}[ht]{0.5\textwidth}
					\centering
					\begin{tabular}{rcc}
						\toprule
						Nome do símbolo & Token & Símbolo\\
						\hline
						\hline
							Ponto e vírgula & sb\_pv & ; \\
							Ponto final &  sb\_pf & . \\ 
							Dois pontos & sb\_dp & : \\ 
							Vírgula & sb\_vg & , \\
							Abre parênteses &sb\_po & ( \\ 
							Fecha parênteses & sb\_pc & ) \\ 
						\bottomrule
					\end{tabular}
					\caption{Quadro de símbolos}
					\label{table:simb}

					\vspace{5mm}
					
					\begin{tabular}{cccc}
						\toprule
						program & begin & end & const \\
						var & real & integer& char\\
						procedure & if & else & readln \\
						writeln & repeat & then & until\\
						while & function & do & \\
						\bottomrule
					\end{tabular}
					\caption{Quadro de palavras reservadas}
					\label{table:reserv}
				\end{minipage}
		
			\end{table}

		\subsubsection{Regras}
			As regras de produção utilizadas são as mesmas regras disponilizadas na LALG. Elas são declaradas utilizando a sintaxe do yacc da seguinte forma:
			\begin{center}
				\begin{minipage}[ht]{0.5\textwidth}
					\begin{verbatim}
						<Não terminal>  :    regra_de_derivação_1
						                |    regra_de_derivação_2
						                |    . . .
						                |    regra_de_derivação_k
						                ;
					\end{verbatim}
				\end{minipage}
			\end{center}

			Observe que dentro das regras de derivações são colocados os tratamentos de erros. Tais tratamentos serão explicados na seção \ref{section:error}
			
		\subsubsection{Programa}
			Nessa seção são implementados o tratamento de erros, que será tratado mais adiante; e a função principal. Nessa ocorre a inicialização da Trie e inserção das palavras reservadas na estrutura gerada.

	\subsection{Tratamento de erros} \label{section:error}
		O analisador sintático identifica corretamente qualquer erro sintático. 
		A partir da identificação de um erro, regras de erro são aplicadas para prosseguir a compilação.

		Os erros são recuperados pela ação default do \emph{YACC}, que é assumir que um erro foi recuperado ao casar três tokens e por meio da chamada a \emph{yyerrok}, contida na regra artificial \verb=ok=. Foi utilizada, também, a função \verb=yyless(0)=, contida na regra artificial \verb=less=, que retorna a última cadeia lida, para sincronização.
		As mensagens de erro são obtidas com maior precisão no \emph{YACC} ao utilizar o modo \emph{verbose}. O modo \emph{verbose}, ao realizar a chamada à função \verb=yyerror=, default ao ocorrer um erro, passa como parâmetro uma cadeia de caracteres contendo o token obtido e a lista de tokens esperados.

		Em linhas gerais, os erros que são recuperados pelo analisador sintático são os seguintes:
		
		\begin{itemize}
			\item Erro na declaração de constantes
			\item Erro na declaração de variáveis
			\item Erro na declaração de procedimentos
			\item Declaração desordenada 
			\item Falta de ponto-vírgula\footnote{Em grande parte dos casos}
			\item Falta de identificador
			\item Símbolos não esperados
			\item Comentário não fechado
			\item Fim de programa inexperado
		\end{itemize}

		Os erros são recuperados no modo pânico, ou seja, tokens são descartados até que se encontre uma regra segura o suficiente para recomeçar a análise sintática.	

		Todos as regras de erros estão comentados nos códigos fonte.


\section{Compilação}\label{section:compilacao}
	Para compilação do projeto em sistemas \emph{UNIX}, execute o arquivo \verb=Makefile= disponibilizado por meio do comando \verb=make=. Para o sistema operacional \emph{Windows}, execute a seguinte sequência de comandos:

	\begin{center}
		\begin{minipage}[ht]{0.5\textwidth}
			\begin{verbatim}
				yacc -d -v yacc.yacc --verbose 
				flex lalg.l
				gcc lex.yy.c trie.c -lfl -o t2.exe
			\end{verbatim}
		\end{minipage}
	\end{center}

	Para execução do analisador sintático utilize os comandos:

	\begin{center}
		\begin{minipage}[ht]{0.5\textwidth}
			\begin{verbatim}
				t2.exe <nome_do_arquivo>
			\end{verbatim}
		\end{minipage}
	\end{center}

	Observe que o analisador só termina quando encontra um \emph{end-of-file}.

	
\section{Exemplos}\label{section:exemplos}
	
	\subsection{Teste 1}
	\subsubsection*{Entrada}

	\begin{lstlisting}[language=Pascal,numbers=left,frame=top,frame=bottom]
program testecerto;
{ declaracao de constantes }
const constante := 3;
const constante2 := 3.57;
{ declaracao de variaveis }
var variavel : integer;
var variavel2, variavel4 : real;
{ declaracao de procedimentos }
{ procedimento sem parametros }
procedure procedimento;
var variavel3 : char;
begin
	readln(variavel);
	if variavel > 3 then
		begin
			variavel2 := constante2;
			variavel := constante;
		end
	else
		while variavel < 3 do
			begin
				variavel := variavel + (variavel + 2)/3;
				variavel := constante2/3 +  variavel;
			end;
	writeln (variavel3);
end;
{ procedimento com parametros }
procedure procedimento2 (a,b: integer; c:real);
var variavel3: integer;
begin
	repeat
		begin
			variavel3 := a+b;
			c := variavel3-1;
		end;
	until variavel3 > 10;
end;
{ inicio do programa }
begin
	procedimento;
	procedimento2 (variavel; variavel; variavel2);
	variavel := variavel * (variavel2 + 3*variavel) / 5;
end.
	\end{lstlisting}
	\subsubsection*{Saída}
		Não há saída por se tratar de um programa sintaticamente correto.

	\subsection{Teste 2}
	\subsubsection*{Entrada}

	\begin{lstlisting}[language=Pascal,numbers=left,frame=top,frame=bottom]
{ teste de erros basicos }
program SemPontoEVirgula { falta ';' }
{ declaracao de constantes }
const constante := 3 { falta ';' }
var variavel2:char{ falta ';' }
var variavel:integer;
{ declaracao de procedimento }
procedure 1erro (param1:integer) { identificador invalido }
var variavelLocal:char;
begin
	readln(variavelLocal) { falta ';' }
	writeln (variavelLocal);
end;
procedure SPtEVgNoFim;
begin
	{ nao faz nada }
end { falta ';' }
{ inicio do programa }
begin
	SPtEVgNoFim { falta ';' }
	variavel := 3*7+12;
	variavel := 3*; { expressao errada }
end.

resto { arquivo nao acabou }
	\end{lstlisting}

	\subsubsection*{Saída}

	\begin{lstlisting}[numbers=left,frame=top,frame=bottom]
Erro na linha 4: 'const' inesperado, esperava ';'
Erro na linha 5: 'var' inesperado, esperava ';'
Erro na linha 6: 'var' inesperado, esperava ';'
Erro na linha 8: 'Numero_mal_formado' inesperado [1erro], esperava 
					'identificador'
Erro na linha 12: 'writeln' inesperado, esperava ';'
Erro na linha 19: 'begin' inesperado, esperava ';'
Erro na linha 21: 'identificador' inesperado [variavel], esperava ';'
Erro na linha 22: ';' inesperado, esperava 'identificador' ou '(' ou 
					'numero_inteiro' ou 'numero_real'
Erro na linha 25: 'identificador' inesperado [resto], esperava 
					'Fim_de_arquivo'
	\end{lstlisting}

	\subsection{Teste 3}

	\subsubsection*{Entrada}

	\begin{lstlisting}[language=Pascal,frame=top,frame=bottom,numbers=left]
program prog;
{ teste de comentario nao fechado
var x: integer;
begin
	readln(x);
end;
	\end{lstlisting}

	\subsubsection*{Saída}

	\begin{lstlisting}[frame=top,frame=bottom,numbers=left]
Erro na linha 2: 'Comentario_nao_fechado'
	\end{lstlisting}
	

\end{document}

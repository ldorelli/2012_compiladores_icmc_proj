\documentclass {article}
\usepackage[margin=1in]{geometry}
\usepackage[brazil]{babel}
\usepackage[utf8]{inputenc}
\usepackage{amsmath}
\usepackage{indentfirst}
\usepackage{cite}
\usepackage{multirow}
\usepackage{verbatim}
\usepackage{listings}
\usepackage{booktabs}
\usepackage{subfig}
\usepackage{graphics}
\usepackage{graphicx}

\title {Trabalho 2}
\author {Bianca Madoka Shimizu Oe \\ 
		Luís Fernando Dorelli de Abreu \\  
		Rafael Umino Nakanishi}

\begin{document}

\titlepage
\maketitle

\section{Objetivo}
	Neste relatório descreveremos a utilização do analisador sintático para a LALG. Apresentaremos as pricipais decisões de projetos e as respectivas justificativas. A seguir uma breve explicação sobre o analisador utilizado e alguns exemplos que foram usados para fins de teste.

	Esse relátório está organizado da seguinte forma. Na Seção \ref{section:lex} são discutidas as mudanças no analisador léxico implementadas para fornecer suporte ao analisador sintático. Na Seção \ref{section:decisoes} são apresentadas as decições de projeto e detalhes de implementação relacionadas ao analisador sintático produzido. Na Seção \ref{section:compilacao} encontram-se as instruções para compilação e execução do analisador sintático. Por fim, na Seção \ref{section:exemplos} são apresentados exemplos de entrada (programas LALG possivelmente incorretos) e suas respectivas saídas.

\section{Mudanças Realizadas}\label{section:lex}

	Para dar suporte ao analisador sintático, foi adicionado ao analisador léxico da etapa anterior um contador de linhas, de forma que, quando um erro sintático for encontrado, seja possível mostrar a linha exata onde o erro ocorreu. A contagem de linhas é independente da presença de comentários: as linhas de comentário também são computadas.

	Também em relação a entrega anterior, foi adicionada a análise léxica a detecção de comentários não fechados. Nesse caso, a contagem de linhas é parada, para que, quando o erro for detectado, seja mostrada a primeira linha do comentário, e não a última. 

\section{Decisões de Projeto}\label{section:decisoes}
	\subsection{Implementação}
		Para implementação de um analisador sintático para a linguagem LALG, optamos pela utilização do YACC (\emph{Yet Another Compiler Compiler}). Essa decisão foi influenciada pela simplicidade e elegância habilitadas pela ferramenta, especialmente no que diz respeito a descrição da gramática.  

		A implementação do analisador supracitado é dividida em três seções: declarações, regras e programas, separadas por dois símbolos de porcentagem (\verb=%%=). A seguir são discutidos o conteúdo implementado em cada seção. 


		\subsubsection{Declarações}
			Nessa seção são declarados todos os tokes utilizados pelo analisador sintático. Todos os tokens declarados são importados da fase anterior do projeto e podem ser vistas nos Quadros \ref{table:op}, \ref{table:simb} e \ref{table:reserv}. 

			Para declaração utilizamos o recurso do YACC \verb=%token=, seguido dos nomes dos tokes e um nome semântico para os mesmos. Esse nome semântico é substituído nas mensagens de erros para uma apresentação mais amigável ao usuário.

			\begin{table}
				\begin{minipage}[ht]{0.5\textwidth}
					\centering
					\begin{tabular}{rcc}
						\toprule
						Nome do operador & Token & Símbolo\\
						\hline
						\hline
							Atribuição & op\_at & := \\
							Igualdade & op\_eq & = \\ 
							Diferença &  op\_df & $<>$ \\ 
							Maior ou igual & op\_ge & $>=$ \\ 
							Menor ou igual & op\_le & $<=$ \\ 
							Maior & op\_gr & $>$ \\
							Menor &op\_ls & $<$ \\ 
							Soma & op\_pl & + \\ 
							Subtração & op\_mi & - \\ 
							Multiplicação & op\_ml & * \\
							Divisão & op\_dv & / \\
						\bottomrule
					\end{tabular}
					\caption{Quadro de operadores}
					\label{table:op}
				\end{minipage}
				~
				\begin{minipage}[ht]{0.5\textwidth}
					\centering
					\begin{tabular}{rcc}
						\toprule
						Nome do símbolo & Token & Símbolo\\
						\hline
						\hline
							Ponto e vírgula & sb\_pv & ; \\
							Ponto final &  sb\_pf & . \\ 
							Dois pontos & sb\_dp & : \\ 
							Vírgula & sb\_vg & , \\
							Abre parênteses &sb\_po & ( \\ 
							Fecha parênteses & sb\_pc & ) \\ 
						\bottomrule
					\end{tabular}
					\caption{Quadro de símbolos}
					\label{table:simb}

					\vspace{5mm}
					
					\begin{tabular}{cccc}
						\toprule
						program & begin & end & const \\
						var & real & integer& char\\
						procedure & if & else & readln \\
						writeln & repeat & then & until\\
						while & function & & \\
						\bottomrule
					\end{tabular}
					\caption{Quadro de palavras reservadas}
					\label{table:reserv}
				\end{minipage}
		
			\end{table}

		\subsubsection{Regras}
			As regras de produção utilizadas são as mesmas regras disponilizadas na LALG. Elas são declaradas utilizando a sintaxe do yacc da seguinte forma:
			\begin{center}
				\begin{minipage}[ht]{0.5\textwidth}
					\begin{verbatim}
						<Não terminal>  :    regra_de_derivação_1
						                |    regra_de_derivação_2
						                |    . . .
						                |    regra_de_derivação_k
						                ;
					\end{verbatim}
				\end{minipage}
			\end{center}

			Observe que dentro das regras de derivações são colocados os tratamentos de erros. Tais tratamentos serão explicados na seção \ref{section:error}
			
		\subsubsection{Programa}
			Nessa seção são implementados o tratamento de erros, que será tratado mais adiante; e a função principal. Nessa ocorre a inicialização da Trie e inserção das palavras reservadas na estrutura gerada.

	\subsection{Tratamento de erros} \label{section:error}
		Em linhas gerais, os erros que sãp tratados pelo analisador sintático são os seguintes:
		
		\begin{itemize}
			\item Falta de ponto-vírgula
			\item Falta de identificador
			\item Símbolos não esperados
			\item Comentaŕio não fechado
			\item Fim de programa inexperado
		\end{itemize}

		Os erros são recuperados no modo pânico, ou seja, tokens são descartados até que se encontre uma regra segura o suficiente para recomeçar a análise sintática.

\section{Compilação}\label{section:compilacao}
	Para compilação do projeto basta executar o arquivo \verb=Makefile= disponibilizado. Dentro do arquivo estão presentes os seguintes comandos:

	\begin{center}
		\begin{minipage}[ht]{0.5\textwidth}
			\begin{verbatim}
				yacc -d -v yacc.yacc --verbose 
				flex lalg.l
				gcc lex.yy.c trie.c -lfl -o t2
			\end{verbatim}
		\end{minipage}
	\end{center}

	Para execução do compilador basta utilizar os comandos:

	\begin{center}
		\begin{minipage}[ht]{0.5\textwidth}
			\begin{verbatim}
				t2 <nome_do_arquivo>
			\end{verbatim}
		\end{minipage}
	\end{center}
	
\section{Exemplos}\label{section:exemplos}

\end{document}

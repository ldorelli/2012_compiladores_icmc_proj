\documentclass {article}
\usepackage[brazil]{babel}
\usepackage[utf8]{inputenc}
\usepackage{amsmath}
\usepackage{indentfirst}
\usepackage{cite}
\usepackage{multirow}
\usepackage{verbatim}
\usepackage{listings}
\usepackage{booktabs}
\usepackage{float}
\usepackage{subfig}
\usepackage{graphicx}

\title {Trabalho 1}
\author {Bianca Madoka Shimizu Oe \\ 
		Luís Fernando Dorelli de Abreu \\  
		Rafael Umino Nakanishi}

\begin{document}

\titlepage
\maketitle

\section{Objetivo}
	Nesse relatório descrevemos como o usar o analisador léxico FLEX. Nas seções seguintes relatamos as principais decisões de projeto que foram feitas e as justificativas para tais escolhas. Em seguida, uma breve explicação de como utilzar o analisador e por fim alguns exemplos.
	
\section{Decisões de Projeto}
	\subsection{Palavras Reservadas}
		Para palavras reservadas do sistema, foi criado uma \emph{trie}(árvore de prefixos), por ser consideravelmente rápida para buscar por strings.
		
		Cada palavra reservada é retornada com o próprio valor. Na tabela \ref{tb_pr} estão listadas as palavras reservadas do sistema.
		\begin{table}[ht]
		
		\centering
		\begin{tabular}{cccc}
			\toprule
			program & begin & end & const \\
			var & real & integer& char\\
			procedure & if & else & readln \\
			writeln & repeat & then & until\\
			while & func & & \\
			\bottomrule
		\end{tabular}
		\caption{Tabela de operadores}
		\label{tb_pr}
		
		\end{table}
		
	\subsection{Tokens}
		
		Existem cinco tipos de tokens que foram usados: operadores, símbolos, identificadores, números e erros. Cada um desses tokens são retornados como:
		
		\begin{equation}
			<palavra\_reservada> \hspace{3mm} <token> \nonumber
		\end{equation}
		
		Além disso, para cada classe dos tokens foram escolhidos alguns padrões que são descritos nas tabelas \ref{tb_op} e \ref{tb_sb}. Os identificadores são retornados como \emph{ident} e dígitos são descritos como \emph{numero\_inteiro} ou \emph{numero\_real}.
		
		\begin{table}[ht]
		
		\centering
		\begin{tabular}{ccc}
			\toprule
			Nome do operador & Token & Símbolo\\
			\hline
			\hline
				Atribuição & op\_at & := \\
				Igualdade & op\_eq & = \\ 
				Diferença &  op\_df & $<>$ \\ 
				Maior ou igual & op\_ge & $>=$ \\ 
				Menor ou igual & op\_le & $<=$ \\ 
				Maior & op\_gr & $>$ \\
				Menor &op\_ls & $<$ \\ 
				Soma & op\_pl & + \\ 
				Subtração & op\_mi & - \\ 
				Multiplicação & op\_ml & * \\
				Divisão & op\_dv & / \\
			\bottomrule
		\end{tabular}
		\caption{Tabela de operadores}
		\label{tb_op}
		
		\end{table}		\begin{table}[ht]
		
		\centering
		\begin{tabular}{ccc}
			\toprule
			Nome do símbolo & Token & Símbolo\\
			\hline
			\hline
				Ponto e vírgula & sb\_pv & ; \\
				Ponto final &  sb\_pf & . \\ 
				Dois pontos & sb\_dp & : \\ 
				Vírgula & sb\_vg & , \\
				Abre parênteses &sb\_po & ( \\ 
				Fecha parênteses & sb\_pc & ) \\ 
			\bottomrule
		\end{tabular}
		\caption{Tabela de símbolos}
		\label{tb_sb}
		
		\end{table}
		
	\subsection{Comentários}
		Para análise de comentários consideramos que eles podem ter mais de uma linha. Com isso, o analisador vai consumindo todos os valores que estiverem entre dos caracteres de comentário( "\emph{\{}" e "\emph{\}}"), inclusive valores de tabulação, espaços vazios e quebra de linhas.
		
	\subsection{Erros}
		Os erros também são retornados com tokens. Os erros que são tratados na análise léxica podem ser vistos na tabela \ref{tb_er}.
		
		\begin{table}[ht]
		\begin{center}
		\begin{tabular}{p{5cm}cp{5cm}}
			\toprule
			Nome do Erro & Token & Causa\\
			\hline
			\hline
				Número mal formado & er\_nmf & Dígitos e caracteres concatenados. Ex.: 12a34b5\\
				Caractere inexistente & er\_cin & Caracteres que não são dígitos, letras, operadores ou símbolos\\
				Tamanho de variável muito grande & er\_idg & Identificador com mais de 20 caracteres\\
				Número muito grande & er\_ifl & Número está fora dos valores máximos e mínimos\\
			\bottomrule
		\end{tabular}
		\caption{Tabela de erros}
		\label{tb_er}
		\end{center}
		\end{table}
	

\section{Compilação}
	
	Em anexo está um arquivo \emph{makefile} que pode ser executado para gerar o executável. Os comandos que estão no arquivo são:
	
	\begin{verbatim}
        flex lalg.l
        gcc lex.yy.c trie.c -lfl -o t1
	\end{verbatim}
	
	Em seguida pode-se utilizar o executável \emph{t1} juntamente com o código fonte escrito em linguagem \emph{lalg}.
	
	\begin{verbatim}
        t1 <arquivo_fonte>
	\end{verbatim}
	
	A saída gerada será o próprio arquivo fonte com os respectivos tokens.
	
\section{Exemplo}

    \subsection{Exemplo 1}
	\begin{verbatim}
        program lalg;
        {entrada}
        var a: integer; 
        begin
        readd(a, @, 1); 
        1.000
        1a222
        1a.000
        1.00a0
        a1ksjdhfgkjsdhfgesuhrguhsfkushdfkguh = 2222222222222222222222222222222222222;
        x <= a;
        ?

        end.
	\end{verbatim}
	
	Saída:
	\begin{verbatim}
        program program
        lalg ident
        ; sb_pv
        var var
        a ident
        : sb_dp
        integer integer
        ; sb_pv
        begin begin
        readd ident
        ( sb_po
        a ident
        , sb_vg
        @ er_cin
        , sb_vg
        1 nro_inteiro
        ) sb_pc
        ; sb_pv
        1.000 nro_real
        1a222 er_nmf
        1a.000 er_nmf
        1.00a0 er_nmf
        a1ksjdhfgkjsdhfgesuhrguhsfkushdfkguh er_idg
        = op_eq
        2222222222222222222222222222222222222 er_ifl
        ; sb_pv
        x ident
        <= op_le
        a ident
        ; sb_pv
        ? er_cin
        end end
        . sb_pf

	\end{verbatim}
	
    \subsection{Exemplo 2}
    
    \begin{verbatim}
        program lalg;
        {isto aqui eh
        um comentario
        de entrada}
        var a: integer;
        begin
        readd(a, @, 1a.31);
        end
    \end{verbatim}
    
    Saida:
    
    \begin{verbatim}
        program program
        lalg ident
        ; sb_pv
        var var
        a ident
        : sb_dp
        integer integer
        ; sb_pv
        begin begin
        readd ident
        ( sb_po
        a ident
        , sb_vg
        @ er_cin
        , sb_vg
        1a.31 er_nmf
        ) sb_pc
        ; sb_pv
        end end

    \end{verbatim}
    
    
    
\end{document}
